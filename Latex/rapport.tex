\documentclass{article}

\usepackage[utf8]{inputenc}
\usepackage[T1]{fontenc}
\usepackage[frenchb]{babel}
\usepackage{amsmath,amsfonts,amssymb,amsthm}
\usepackage[margin=1in]{geometry}
\usepackage{graphicx}
\usepackage{hyperref}

\setcounter{secnumdepth}{0}
\graphicspath{ {./Images/} }

\begin{document}

\begin{titlepage}

	\begin{center}
		\hrule

		\vspace{.5cm}

		\Huge
		\textbf{Téléinformatique}

		\vspace{.3cm}
		\LARGE

		\textbf{IFT 3325}
		\vspace{.3cm}

		\textbf{Devoir n°3}
		\vspace{.3cm}

		\hrule

		\vspace{1cm}

		17 Décembre 2023 \\
	\end{center}

	\vspace{2cm}

	\LARGE

	\noindent Auteurs :

	\begin{enumerate}
		\item[-] Léo Jetzer (?)
		\item[-] Luchino Allix-Lastrego (20222844)   
	\end{enumerate}


			
	\vfill


	\begin{center}

		\includegraphics[scale=.1]{diro.png}

		\vspace{0.8cm}

		Université de Montréal\\
		Département d'informatique et de recherche opérationnelle\\

	\end{center}
	
\end{titlepage}

\section{Exercice 1 \emph{(10 points)}}

\subsection{a. \emph{(6 points)}}

\subsection{b. \emph{(4 points)}}

\clearpage

\section{Exercice 2 \emph{(16 points)}}

\clearpage

\section{Exercice 3 \emph{(12 points)}}

\subsection{a. \emph{(6 points)}}

\subsection{b. \emph{(6 points)}}

\clearpage

\section{Exercice 4 \emph{(10 points)}}

\subsection{a. \emph{(3 points)}}

\subsection{b. \emph{(4 points)}}

\subsection{c. \emph{(3 points)}}

\clearpage

\section{Exercice 5 \emph{(12 points)}}

\clearpage

\section{Exercice 6 \emph{(12 points)}}

\clearpage

\section{Exercice 7 \emph{(7 points)}}

\clearpage

\section{Exercice 8 \emph{(10 points)}}

\clearpage

\section{Exercice 9 \emph{(6 points)}}

\clearpage

\section{Exercice 10 \emph{(5 points)}}


\end{document}
