\documentclass{article}

\usepackage[utf8]{inputenc}
\usepackage[T1]{fontenc}
\usepackage[frenchb]{babel}
\usepackage{amsmath,amsfonts,amssymb,amsthm}
\usepackage[margin=1in]{geometry}
\usepackage{graphicx}
\usepackage{hyperref}

\setcounter{secnumdepth}{0}
\graphicspath{ {./Images/} }

\begin{document}

\begin{titlepage}

	\begin{center}
		\hrule

		\vspace{.5cm}

		\Huge
		\textbf{Téléinformatique}

		\vspace{.3cm}
		\LARGE

		\textbf{IFT 3325}
		\vspace{.3cm}

		\textbf{Devoir n°3}
		\vspace{.3cm}

		\hrule

		\vspace{1cm}

		17 Décembre 2023 \\
	\end{center}

	\vspace{2cm}

	\LARGE

	\noindent Auteurs :

	\begin{enumerate}
		\item[-] Léo Jetzer (20070432)
		\item[-] Luchino Allix-Lastrego (20222844)   
	\end{enumerate}


			
	\vfill


	\begin{center}

		\includegraphics[scale=.1]{diro.png}

		\vspace{0.8cm}

		Université de Montréal\\
		Département d'informatique et de recherche opérationnelle\\

	\end{center}
	
\end{titlepage}

\section{Exercice 1 \emph{(10 points)}}

\subsection{a. \emph{(6 points)}} % A developper

Pour réduire la charge d'un serveur il faut favoriser l'utilisation de Go-Back-N car les acknowledge cumulatifs permettent de diminuer le nombre de réponses par rapport au grand nombre de petits messages.

\subsection{b. \emph{(4 points)}}

\clearpage

\section{Exercice 2 \emph{(16 points)}}

\subsection{Dijkstra}

Dans le tableau ci-dessous, les colonnes indiquent les sommets et les lignes le sommet où l'on est actuellement. Par exemple \emph{E (5)} signifie qu'on se trouve sur le sommet E et que le poid associé pour arriver à ce sommet est de 5. Le croisement entre une ligne et une colonne indique comment faire pour arriver à ce sommet. Par exemple, \emph{5 B} à la ligne \emph{H (4)} et à la colonne \emph{C} indique que pour se rendre en \emph{C} le plus court chemin vaut 5 et passe par \emph{B}. Le symbole $\infty$ indique que le sommet n'a pas encore pu être atteint et '-' indique que le chemin à déjà été visité. 

\hfill

\begin{tabular}{|c|c|c|c|c|c|c|c|c|c|c|c|}
	\hline
	& A & B & C & D & E & F & G & H & I & J & Z\\
	\hline
	Départ & 0 A & $\infty$ & $\infty$ & $\infty$ & $\infty$ & $\infty$ & $\infty$ & $\infty$ & $\infty$ & $\infty$ & $\infty$\\
	\hline 
	A (0) & - & 3 A & $\infty$ & $\infty$ & 5 A & $\infty$ & $\infty$ & 4 A & $\infty$ & $\infty$ & $\infty$\\
	\hline
	B (3) & - & - & 5 B & $\infty$ & 5 A & 10 B & $\infty$ & 4 A & $\infty$ & $\infty$ & $\infty$\\
	\hline
	H (4) & - & - & 5 B & $\infty$ & 5 A & 9 H & $\infty$ & - & 6 H & $\infty$ & $\infty$\\
	\hline
	E (5) & - & - & 5 B & $\infty$ & - & 9 H & $\infty$ & - & 6 H & $\infty$ & $\infty$\\
	\hline
	C (5) & - & - & - & 8 C & - & 7 C & 11 C & - & 6 H & $\infty$ & $\infty$\\
	\hline
	I (6) & - & - & - & 8 C & - & 7 C & 11 C & - & - & 12 I & $\infty$\\
	\hline
	F (7) & - & - & - & 8 C & - & - & 11 C & - & - & 10 F & $\infty$\\
	\hline
	D (8) & - & - & - & - & - & - & 11 C & - & - & 10 F & 10 D\\
	\hline
\end{tabular}

\hfill

\hfill

À la dernière ligne du tableau, on voit que l'on peut arriver en \emph{Z} en venant de D avec un chemin de poid 10. Ceci met fin à l'algorithme car les autres chemins qui n'arrivent pas encore à \emph{Z} sont de poid supérieur ou égal à 10. Pour retrouver le chemin parcouru on remonte le tableau. On arrive en \emph{Z} depuis \emph{D}, on arrive en \emph{D} depuis \emph{C} et ainsi de suite pour obtenir le chemin  de poid 10 : \emph{ABCDZ}.

\subsection{Bellman-Ford}

Dans le tableau ci dessous, les colonnes indiquent le sommet et les lignes le nombre maximum de chemin que l'on peut prendre pour arriver au sommet. La logique reste la même concerant les cases. Par exemple \emph{5 B} à la ligne \emph{4} et à la colonne \emph{C} indique que pour se rendre en \emph{C} avec au plus 4 chemins emprunté, le plus court chemin vaut 5 et passe par \emph{B}.Comme précedement $\infty$ indique que le sommet n'a pas encore pu être atteint.

\hfill

\begin{tabular}{|c|c|c|c|c|c|c|c|c|c|c|c|}
	\hline
	Itération & A & B & C & D & E & F & G & H & I & J & Z\\
	\hline
	0 & 0 A & $\infty$ & $\infty$ & $\infty$ & $\infty$ & $\infty$ & $\infty$ & $\infty$ & $\infty$ & $\infty$ & $\infty$\\
	\hline 
	1 & 0 A & 3 A & $\infty$ & $\infty$ & 5 A & $\infty$ & $\infty$ & 4 A & $\infty$ & $\infty$ & $\infty$\\
	\hline
	2 & 0 A & 3 A & 5 B & $\infty$ & 5 A & 9 E & $\infty$ & 4 A & 6 H & $\infty$ & $\infty$\\
	\hline
	3 & 0 A & 3 A & 5 B & 8 C & 5 A & 7 C & 11 C & 4 A & 6 H & 12 I & $\infty$\\
	\hline
	4 & 0 A & 3 A & 5 B & 8 C & 5 A & 7 C & 11 C & 4 A & 6 H & 10 F & 10 D\\
	\hline
	5 & 0 A & 3 A & 5 B & 8 C & 5 A & 7 C & 11 C & 4 A & 6 H & 10 F & 10 D\\
	\hline
	6 & 0 A & 3 A & 5 B & 8 C & 5 A & 7 C & 11 C & 4 A & 6 H & 10 F & 10 D\\
	\hline
	7 & 0 A & 3 A & 5 B & 8 C & 5 A & 7 C & 11 C & 4 A & 6 H & 10 F & 10 D\\
	\hline
	8 & 0 A & 3 A & 5 B & 8 C & 5 A & 7 C & 11 C & 4 A & 6 H & 10 F & 10 D\\
	\hline
	9 & 0 A & 3 A & 5 B & 8 C & 5 A & 7 C & 11 C & 4 A & 6 H & 10 F & 10 D\\
	\hline
	10 & 0 A & 3 A & 5 B & 8 C & 5 A & 7 C & 11 C & 4 A & 6 H & 10 F & 10 D\\
	\hline
\end{tabular}

\hfill

\hfill

On remarque que à partir de la ligne 4, plus rien ne change, en effet tous les plus courts chemins depuis le sommet A vers les autres sommets empruntent au plus 4 arrêtes. Pour trouver le chemin le plus court de \emph{A} à \emph{Z} même logique que précedement, ce qui nous donne : \emph{ABCDZ} avec un poid de 10.


\clearpage

\section{Exercice 3 \emph{(12 points)}}

\subsection{a. \emph{(6 points)}} % A developper 

Les deux réseaux ne peuvent pas communiquer car ils ont le même network ID.


\noindent \begin{minipage}{9cm}
	\begin{align*}
		A &= 01100101.01000000.00000000.01100110\\
		M &= 11111111.11111111.00000000.00000000\\
		A \land  M &= 01100101.01000000.00000000.00000000\\
	\end{align*}
\end{minipage}
\begin{minipage}{7cm}
	\begin{align*}
		B &= 01100101.01000000.00101101.01100110\\
		M &= 11111111.11111111.00000000.00000000\\
		B \land  M &= 01100101.01000000.00000000.00000000\\
	\end{align*}
\end{minipage}

\hfill

En effet, on remarque que $A \land  M = B \land  M$

\subsection{b. \emph{(6 points)}}

\clearpage

\section{Exercice 4 \emph{(10 points)}}

\subsection{a. \emph{(3 points)}}

\subsection{b. \emph{(4 points)}}

\subsection{c. \emph{(3 points)}}

\clearpage

\section{Exercice 5 \emph{(12 points)}}

\clearpage

\section{Exercice 6 \emph{(12 points)}}

\clearpage

\section{Exercice 7 \emph{(7 points)}}

\clearpage

\section{Exercice 8 \emph{(10 points)}}

\clearpage

\section{Exercice 9 \emph{(6 points)}}

\clearpage

\section{Exercice 10 \emph{(5 points)}}

Non les algorithmes de Dijkstra et Bellman-Frod ne produisent pas tout le temps les mêmes résultats car l'algorithme de Dijkstra ne permet pas des arrêtes avec des poids négatifs alors que celui de Bellman-Ford oui. Donc dans un graph avec au moins une arrête de poid négatif, les deux algorithmes ne produiront pas peut être pas le même résultat.


\end{document}
